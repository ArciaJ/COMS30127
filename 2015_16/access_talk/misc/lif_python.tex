\documentclass[11pt,a4paper]{scrartcl}
\typearea{12}
\usepackage{graphicx}
\usepackage{pstricks}
\usepackage{listings}
\lstset{language=python}
\pagestyle{headings}
\markright{Leaky integrate and fire neuron}
\begin{document}
\subsection*{The leaky integrate and fire model}

The leaky integrate and fire model is a simple model of the neuron; it
has been used in neuroscience to study large networks of neurons and
to investigate how neuronal networks might learn through changes in
the strengths of synapses. Here we will look at a single neuron. 

Say $V$ is the voltage inside the neuron, the equation for how the
voltage changes is given by
\begin{equation}
\tau_m\frac{dV}{dt}=E_L-V+R_mI
\end{equation}
So
\begin{itemize}
\item $dV/dt$ the rate of change of the voltage.
\item $I$ the current coming in to the cell changing the voltage,
  $R_m$ basically measures how much the charge changes the voltage.
\item $E_L-V$ the charge also leaks out and it leaks out more for
  bigger $V$. $E_L=-70$ mV, the reason this isn't zero is complicated
  and related to entropy.
\item $\tau_m$ is a time constant, usually about 10 ms. It determines
  how fast $V$ responds.
\end{itemize}

That's part of the behavior of the neuron; there is also spiking, that
isn't modelled by the integrate and fire model. Instead there is a
rule that when $V\ge V_T=-55$ mV there's a spike and the voltage is
set back to zero.

\subsection*{Simulating an integrate and fire neuron}

The idea here is to solve the integrate and fire equation on a
computer, the usual way to do that is to divide time into little
slices, say $\delta t$ wide and pretend $dV/dt$ is $\delta V/\delta t$,
the change in $V$ divided by the change in $t$. This gives the so
called Euler method:
\begin{equation}
V\rightarrow V+\delta V
\end{equation}
where 
\begin{equation}
\delta V= \frac{(E_L-V+R_mI)\delta t}{\tau_m}
\end{equation}

\newpage

\subsection*{A Python version}

On the Raspberry Pi you can open a text editor by typing
\texttt{nano}. This is a very simple text editor but it has the
advantage that all the commands are listed at the bottom. $\wedge X$, that
is control-X, exits and will ask if you want to save your program as
you do; there is no other way to save your program. If you want to
edit your existing program rather than start a new one you write
\texttt{nano foo.py} assuming your program is called \texttt{foo.py}. 

Here is a Python program to do an integrate and fire neuron
\begin{lstlisting}[numbers=right]
e_l = -70
tau_m = 10
v_t = -55

r_i = 10

delta_t = 0.1

t=0

big_t=2000

v=e_l

while t<=big_t:
	v+=(e_l-v+r_i)*delta_t/tau_m
	if v>=v_t:
		print "|",
		v=e_l	
	else:
		print "_",

	t+=delta_t
\end{lstlisting}
The numbers on the right aren't part of the program, it is just handy
to have them. Can you work out what the different parts of the program do? To run it, save it, say as \texttt{lif.py} and then type \texttt{python lif.py}. Try changing the input current. Can you write another program to plot the firing rate, that is the average number of spikes a second, as a function of the input?

\end{document}
