\documentclass[11pt,a4paper]{scrartcl}
\typearea{12}
\usepackage{graphicx}
\usepackage{pstricks}
\usepackage{listings}
\lstset{language=python}
\pagestyle{headings}
\markright{Computation Neuroscience - coursework 1}
\begin{document}
\subsection*{Coursework 1}

\subsubsection*{Solving differential equations with Euler's method}

The following questions are to help you practise your Python, they are
not to be submitted and will not be marked, instead you can compare to
the model graphs available in the same github folder.

\begin{enumerate}

\item Solve numerically in Python using Euler's method the differential equation
\begin{equation}
\frac{df}{dt}=f^2-3f+e^{-t}
\end{equation}
on the interval $[0,3]$ with time step $\delta t=0.01$ and graph the
solution, taking care to label the axes. Although Python has good
libraries for solving differential equations numerically it would be
useful educationally not to use them for these question.

\item For the problem above try solve with $\delta t=0.01$, $0.1$,
  $0.5$ and one. Plot all the curves on one graph. What is a good
  value of $\delta t$ for this equation.

\end{enumerate}

\subsubsection*{Integrate and fire neurons}

The questions in this section are to be submitted for marking and will
make up $10\%$ of your final mark. In this and next assignments, all
the plots should have axes labels, and if there are multiple graphs on
a plot, the legend (or key) should be included. For each missing label
or legend, 1$\%$ of mark will be subtracted.

Write a brief report, no longer than two pages or no longer than four
pages if you include the extra part, including the figures and the
comments specified above; submissions exceeding the page limit will be
rejected, I will take a dim view of super-narrow margins or tiny
fonts. Submit it in the pdf format together with the Python code by
the deadline. Remember, provided it is in good time, I am happy to
answer questions about Python and to help debug faulty code.

\begin{enumerate}

\item Simulate an integrate and fire model with the following
  parameters for 1 s: $\tau_m = 10 $ms, $E_L = V_r = -70$ mV, $V_t =
  -40$ mV, $R_m= 10$ M$\Omega$, $I_e = 3.1 $ nA. Use Euler's method
  with timestep $\delta t = 1$ ms. Here $E_L$ is the leak potential,
  $V_r$ is the reset voltage, $V_t$ is the threshold, $R_m$ is the
  resistance and $\tau_m$ is the membrane time constant. Plot the
  voltage as a function of time. For simplicity assume that the neuron
  does not have a refractory period after producing a spike. [20\% of
    marks]. You do not need to plot spikes - once membrane potential
  exceeds threshold, simply set the membrane potential to $V_r$.

\item Compute analytically the minimum current $I_e$ required for the
  neuron with the above parameters to produce an action
  potential. [10\% of marks].

\item Simulate the neuron for 1 s for the input current with amplitude
  $I_e$ which is 0.1 [nA] lower than the minimum current computed
  above, and plot the voltage as a functions of time. [15\% marks].

\item Simulate the neuron for 1s for currents ranging from 2 [nA] to 5
  [nA] in steps of 0.1 [nA]. For each amplitude of current count the
  number of spikes produced, that is the firing rate. Plot the firing
  rate as the function of the input current. [15\% of marks]. It is
  possible to calculate this curve analytically; there is no
  requirement that you do this, but you might find it interesting to
  try.

\item Simulate two neurons which have synaptic connections between each other, i.e. the first neuron projects to the second, and the second neuron projects to the first. Both model neurons should have the same parameters: $\tau_m = 20$ ms, $E_L = -70$ mV $V_r = -80$ mV $V_t = -54$ mV $R_mI_e = 18$ mV and their synapses should also have the same parameters: $R_m \bar{g}_s = 0.15$, $P = 0.5$, $\tau_s= 10$ ms. For simplicity take the synaptic conductance to satisfy
\begin{equation}
\tau_s\frac{dg_s}{dt}=-g_s
\end{equation}
with a spike arriving causing $g_s$ to increase by $\bar{g}_sP$. Simulate two cases: a) assuming that the synapses are excitatory with $E_s = 0$ mV, and b) assuming that the synapses are inhibitory with $E_s = -80$ mV. For each simulation set the initial membrane potentials of the neurons $V$ to different values chosen randomly from between $V_r$ and $V_t$ and simulate 1 s of activity. For each case plot the voltages of the two neurons on the same graph (with different colours). [20\% of marks].

\item In many real neurons the firing rate falls off after the first few spikes. This can be simulated with a slow potassium current. For the neuron described in the first question add a slow potassium current. This current should have reversal potential $E_K=-80$ mV, its conductance should increase by 0.005 $($M$\Omega)^{-1}$ every time there is a spike, otherwise it should decay towards zero with time constant $\tau=200$ ms. Plot the voltage of this neuron for one second. [10\% of marks].

\item This is an extra question, not really worth the ten percent of
  marks for this assignment that are on offer for it; it is more to do
  if you are feeling keen. Simulate a Hodgkin-Huxley neuron for
  different input currents and use some graphs to illustrate your
  results. Feel free to use the bits of code on the course
  website. [10\% of marks].

\end{enumerate}

\end{document}

