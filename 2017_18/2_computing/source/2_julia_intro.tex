\documentclass[11pt,a4paper]{scrartcl}
\typearea{12}
\usepackage{graphicx}
\usepackage{pstricks}
\usepackage{listings}
\pagestyle{headings}
\newcommand{\lnn}[1]{\textbf{line #1}\,}
\newcommand{\Lnn}[1]{\textbf{Line #1}\,}
\markright{Julia introduction}
\lstdefinelanguage{Julia}%
  {morekeywords={abstract,break,case,catch,const,continue,do,else,elseif,%
      end,export,false,for,function,immutable,import,importall,if,in,%
      macro,module,otherwise,quote,return,switch,true,try,type,typealias,%
      using,while},%
   sensitive=true,%
%   alsoother={$},%
   morecomment=[l]\#,%
   morecomment=[n]{\#=}{=\#},%
   morestring=[s]{"}{"},%
   morestring=[m]{'}{'},%
}[keywords,comments,strings]%
\lstset{%
    language         = Julia,
    basicstyle       = \ttfamily,
    keywordstyle     = \bfseries\color{blue},
    stringstyle      = \color{magenta},
    commentstyle     = \color{ForestGreen},
    showstringspaces = false,
}

\begin{document}
\section*{Julia Introduction\footnote{\texttt{https://github.com/conorhoughton/teaching\_misc/tree/master/python\_workshop/}}}
\subsection*{Introduction}
Julia is a new programming language that runs much faster than Python;
it is possible to write fast code in MATLAB where everything is
written as linear algebra or using carefully optimized numpy or fancy
just-in-time compilation in Python; the idea of Julia is that it runs
at C-like speeds for natural, modern-looking code. It does this by
allowing, while not requiring, type declaration and by not having
classes, instead it has \textsl{types}, a bit like \textsl{structs} in
C, and multiple dispatch. 

It has other features to make it useful for scientific computing,
little things like being able to \texttt{2v} when you mean
\texttt{2*v}, along with big things, like a sophisticated
multi-dimensional array datatype that can be used for efficient matrix
operations. Presumably to help persuade people to finally abandon
MATLAB it has a MATLAB-like syntax, blocks are denoated using a
\texttt{end} keyword, the first element of an array \texttt{a} is
\texttt{a[1]} and \texttt{1:10} means one to ten, not one to nine.

If you are used to Python Julia can seem frustrating to debug, mostly
because the typing can be hard to get used to, but debugging Julia as
a Python programmer really reminds you how often you cast variables
without even noticing; this is part of why Julia is much faster.

This only outlines the simplest parts of the language, the wikibook
\begin{quote}
https://en.wikibooks.org/wiki/Introducing\_Julia/
\end{quote}
is a good place to look for a longer introduction. There is online Julia at
\begin{quote}
https://juliabox.com/
\end{quote}

\subsection*{A simple example}

Here is a programme to add powers of two (\texttt{add.jl}):
\begin{lstlisting}[numbers=left]
highest_power=10

value=1.0::Float64
current=0.5::Float64

for i in 1:highest_power
   value+=current
   current*=0.5
end

println(value)
\end{lstlisting}
\Lnn{1} defined \texttt{highest\_power}; this is dynamically typed, as
in Python, but \texttt{value} and \texttt{current} are given a type,
\texttt{Float64}; as an indication of how seriously it takes typing,
is \lnn{4} was \texttt{value=1::Float64} it would return an error
since \texttt{1} isn't a \texttt{Float64}. You can find a full list of
types in the wikibook, it has lots of different int and float types,
along with rational numbers using \texttt{//} to seperate numerator and divisor (\texttt{rational.jl}):
\begin{lstlisting}[numbers=left]
a=2//3
b=1//2
println(a-b)
\end{lstlisting}

\subsection*{Arrays}
Arrays are what Python calls lists, python is the odd one out here,
array is a more common name. Julia has the same slicing functionality
as Python, although as mentioned above indexing is different (\texttt{slice.jl})
\begin{lstlisting}[numbers=left]
a=[1,2,3,4,5]
println(a[1:3])

for i in a
   println(i)
end

\end{lstlisting}
prints \texttt{[1,2,3]} from \lnn{2}, \lnn{4} to \lnn{6} demonstrates
a for loop. The last element in an array is indexed \texttt{end} so in
the programme above \texttt{a[end]} is \texttt{5}.

Arrays can store mixed items, but the array can be typed, so \texttt{a} in (\texttt{typed\_list.jl}
\begin{lstlisting}[numbers=left]
a=Int64[1,2,3,4,5]
push!(a,6)
println(a)
\end{lstlisting}
can only store items of type \texttt{Int64}. \texttt{push!} pushes an
item onto the list, like \texttt{append} in Python, again, this is the
more common notation. The \texttt{!} is part of an convention where
all commands that change an array have a \texttt{!}.

As mentioned above, Julia arrays can be multidimensional and have
matrix like operations, but this won't be explored in this brief
overview. There is also a tuple type which is immutable.

\subsection*{Functions}
Here is a programme with some functions (\texttt{functions.jl})
\begin{lstlisting}[numbers=left]
function add_to_int(a::Integer,b::Integer)
	 println("int version")
	 a+b
end

function add_to_int(a::Real,b::Real)
	 println("float version")
	 convert(Int64,a+b)
end

function add_to_int(a,b)
	 println("what are these things")
	 0
end

println(add_to_int(12,6))
println(add_to_int(12.0,6.0))
\end{lstlisting}
Obviously this is a very artificial example, but it shows some of the
features of functions, first, their return value is the most recently
evaluated expression and second, and more importantly, they support
multiple dispach; the function is chosen to match the type of the
arguments, here there is one function for \texttt{Integers}, this is a
supertype which includes, for example, \texttt{Int64}, there is one
for \texttt{Real}, the supertype that includes various floats, and one
with no type; the correct function is used for each. If there is no
correct fuction there will be an error.

\subsection*{Composite Types}
Julia doesn't have classes, this comes as a surprise at first, but it
does have composite data types that work like structs, combining this
with multiple dispach captures important parts of the functionality of
classes, in a way that supports fast code (\texttt{struct\_example.jl}
\begin{lstlisting}[numbers=left]
mutable struct Cow
	name::String
	age::Int64
end

mutable struct Poem
	name::String
end	

function move(cow::Cow)
	 print(cow.name," walks forward") 
         println("showing the weight of her ",cow.age," years")
end

function move(poem::Poem)
	 println(poem.name, " moves us to tears with its beauty")
end

poem = Poem("The Red Wheelbarrow")
cow = Cow("Hellcow",42)

move(cow)
move(poem)
\end{lstlisting}
You can see that although the structs have no methods, the function
\texttt{update} can have different meaning for the two different data
types. The default constructor defines the variables in the order they
appear, it is possible to define other constructors, but that won't be
considered here.

\end{document}

