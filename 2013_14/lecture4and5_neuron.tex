\documentclass[11pt,a4paper]{scrartcl}
\typearea{12}
\usepackage{graphicx}
\usepackage{pstricks}
\usepackage{listings}
\lstset{language=python}
\pagestyle{headings}
\markright{Computation Neuroscience - Lecture 2}
\begin{document}

\subsection*{Introduction}
These notes are about the dynamics of a single neuron, it will cover
the Hodgkin Huxley equation, the Integrate and Fire model and the
behavior of synapses. I offer an error bounty of between 20p and 2
pounds for mistakes. Contact me at
\texttt{conor.houghton@bristol.ac.uk} or come up after a lecture.

\subsection*{Electrical properties of a neuron}
The potential inside a neuron is lower than the potential on the
outside; this difference is created by ion pumps, small molecular
machines that use energy to pump ions across the membrane seperating
the inside and outside of the cell. One typical ion pump is
Na+/K+-ATPase (Sodium-potassium adenosine triphosphatase); this uses
energy in the form of ATP, the energy carrying molecule in the body,
and through each cycle, it moves three sodium ions out of the cell and
two potassium ions into the cell. Since both sodium and potassium ions
have a charge of plus one, this leads to a net loss of one atomic
charge to the inside of the cell lowering its potential. It also
creates an excess of sodium outside the cell and an excess of
potassium inside it. We will return to these chemical imbalances
later. The potential difference across the membrane is called the
\textbf{membrane potential}. At rest a typical value of the membrane
potential is $E_L=-70 $mV.

There is an interesting argument that explains the voltage scales for
the electrodynamics of neurons; it is useful because it touches on
themes we will return to in our study of neurons. Basically we will
see that neurons work partly due to diffusion, which in turn depends
on ions fly around because of their thermal energy. We know the
thermal energy of an ion at temperature $T$, it is $k_BT$ where $k_B$
is the Boltzmann constant. Now, consider the potential difference with
that corresponding energy, the energy required to move an ion of
charge one across a voltage $V_0$ is $qV_0$, so for neurons to work we
would expect the scale of the voltages involve to be of the order
where the thermal energy was similar to the energy required to
overcome the voltage gap, that is, we expect the voltage gap to be able to modulate that flow. Hence $qV_0\approx k_BT$ or
\begin{equation}
V_0\approx \frac{k_BT}{q}\approx 27\,\mbox{mV}
\end{equation}
at room temperature.

\subsection*{Spikes}

So the summary is that \textbf{synapses} cause a small increase or
decrease in the voltage; \textbf{excitatory synapses} cause an
increase, \textbf{inhibitory synapses} a decrease. This drives the
internal voltage dynamics of the cell, these dynamics are what we will
learn about here. If the voltage exceeds a threshold, say $V_T=-55$ mV
there is a nonlinear cascade which produces a \textbf{spike} or
\textbf{action potential}, a spike in voltage 1-2 ms wide which rises
above 0 mV before, in the usual description, falling to a reset value
of $V_R=-65$ mV, the cell then remains unable to produce another spike
for a \textbf{refractory period} which may last about 5 ms.

\subsection*{Buckets of water}

In the simplest model of neurons their voltage dynamics is similar to
the dynamics of a bucket with a leak. In this analogy a bucket, with
straight sides, is filled to a height $V$, water pours in the top at a
rate $I$, which might depend on time, so $I(t)$, and water leaks out a hole
in the bottom. The amount of water leaking out is $GV$, $V$ bacause
the more water there is, the higher the pressure at the hole and $G$
so that we can specify the size of the hole.

Now, the $V$ is the height of the water not the volume, the confusing
use of $V$ is to aid the analogy with neurons where $V$ is the
voltage. The volume is $VC$ where $C$ is the cross-sectional area of the bucket. The rate of change of the volume is the water flowing in $I$, hence
\begin{equation}
\frac{dCV}{dt}=I-GV
\end{equation}
or
\begin{equation}
\frac{dV}{dt}=\frac{1}{C}(I-GV)
\end{equation}

Lets solve this equation for constant $I$ before going on to look at
neurons. Probably best to do this using an integrating factor, let
$\tau=C/G$ and $\tilde{I}=I/G$
\begin{equation}
\tau\frac{dV}{dt}+V=\tilde{I}
\end{equation}
then we multiply across by $\exp{t/tau}$
\begin{equation}
\tau e^{t/\tau}\frac{dV}{dt}+e^{t/tau}V=\tilde{I}e^{t/tau}
\end{equation}
Now we can rewrite the left hand side using the product rule
\begin{equation}
\frac{d}{dx}(uv)=u\frac{dv}{dx}+v\frac{du}{dx}
\end{equation}
to give
\begin{equation}
\tau\frac{d}{dt}\left(e^{t/\tau}V\right)=\tilde{I}e^{t/tau}
\end{equation}
Now integrating both sides gives
\begin{equation}
e^{t/\tau}V=\tilde{I}e^{t/tau}+A
\end{equation}
where $A$ is an integration constant. This gives
\begin{equation}
V=Ae^{-t/\tau}+\tilde{I}
\end{equation}
and putting $t=0$ shows $A=V(0)-\tilde{I}$ so
\begin{equation}
V=[V(0)-\tilde{I}]e^{-t/\tau}+\tilde{I}
\end{equation}
so, basically, the value of $V$ decays exponentially until it equilibriates with $\tilde{I}$.

\subsection*{The bucket-like equation for neurons}

We will now try to extend this equation so that it applies to
neurons. First off $V$ is now voltage and $C$ will be replaced by
$c_m$, the capacitance of the membrane, the amount of electrical
charge that can be stored at the membrane is $C_mV$, the amount of
electrical charge is the analog of the volume of water. The charge
leak is a bit more complicated, because of the chemical gradients,
that is the effects of the differing levels of ions inside and outside
the cell along and their propensity to diffuse, the voltage at which
there is no leaking of charge is not zero, it is $E_L=-70 $mV,
roughly.  $G$ is now $G_m$, a conductance, the leak current
is $G_m(V-E_L)$, as above, we actually divide across by it, and write
$R_m=1/G_m$, the resistance. Finally, we write $\tau_m=C_m/G_m$ to get
\begin{equation}
\tau_m\frac{dV}{dt}=E_L-V+R_mI
\end{equation}
$I$ might end up being synaptic input, but traditionally we write the
equation to match the \textsl{in vivo} experiment where $I$ is an
injected current from an electrode, so we write $I_e$, \lq{}e\rq{} for
electrode.

This leaves out the possibility that there are other non-linear
changes in the currents through the membrane as $V$ changes. This is a
problem since there are other non-linear changes in the currents
through the membrane as $V$ changes. The equation above leaves these
out, in fact, the nonlinear effects are strongest for values of $V$
near where a spike is produced, so one approach is to use the linear
equation unless $V$ reaches a threshold value and then add a spike \lq{}by
hand\rq{}. This is the \textbf{leaky integrate and fire model}.

\begin{itemize}
\item $V$ satisfies
\begin{equation}
\tau_m\frac{dV}{dt}=E_L-V+R_mI_e
\end{equation}
\item If $V\ge V_T$ a spike is recorded and the voltage is set to $V_R$.
\end{itemize}

This model is easy to solve; if $I_e$ is constant we have already
solved it above up to messing around with constants. If $I_e$ is not
constant it may still be possible to solve the equation, but in any
case the equation can be solved numerically on a computer. Basically
you divide time up into small steps, assume $I_e$ is constant for each
small step and use the analytic constant-input equation to get from
one time step to another. Alternatively the equation can be solved
using a numerical approach to solving differential equations, such as
the Euler method, or Runge Kutta.
\end{document}

