\documentclass[11pt,a4paper]{scrartcl}
\typearea{12}
\usepackage{graphicx}
\usepackage{pstricks}
\usepackage{listings}
\lstset{language=python}
\pagestyle{headings}
\markright{Computation Neuroscience - 2 python quick tour}
\begin{document}
\subsection*{Introduction}
These are getting started notes on Python, the name of the
corresponding example program is given in brackets in a fixed width
font (\texttt{fixed width font}) just before the code listing.
\subsection*{Computing in Computational Neuroscience}
People in computational neuroscience usually use either MATLAB or
Python or they use a specialized simulation tool such as GENESIS or
NEURON or NEST. These simulation tools are optimized to efficiently
simulate large complicated neurons, in the case of GENESIS or NEURON,
or large and complicated networks of neurons, in the case of NEST. The
simulation tools have been writen over many years and can be quite
complicated and idiosyncratic, however, NEURON at least, now has a
Python interface. 

MATLAB has some advantages, it has a very large user base across many
parts of applied mathematics, a huge collection of libraries and
package. There is a free community authored version called Octave,
they are not completely compatible and Octave lacks many libraries and
features. Matlab uses a matrix paradigm which is easy to use and quick
when you get used to it. Python is a proper programming language, with
a nicer language structure than Matlab, it is free and open and has
many libraries, though perhaps not as many as Matlab.

Other commonly used tools include Brian, a Python based package which
supplies efficient specification and integration of differential
equations used in neuronal simulations, XPP which is useful for
analysing the dynamics of the differential equations used in
neuroscience, NeuroML which is a meta-language for describing neuronal
models and R which is used for statistics.

These notes are only the roughest of introductions to Python, remember
google and stackoverflow are your friends.

\subsection*{Python - getting started}
Python either runs as a script or on an interpreter. To get the
interpreter you type \texttt{python} on the command line and you get
something that looks like this:
\begin{lstlisting}[numbers=right]
Python 2.7.6 (default, Jun 22 2015, 17:58:13) 
[GCC 4.8.2] on linux2
Type "help", "copyright", "credits" or "license" for more information.
>>> 
>>> 
>>> 
\end{lstlisting}
where the numbers on the right are just for these notes. The
\lq{}2.7.6\rq{} is the version number; one annoying thing about Python
is that Python 3 and higher are not fully backwards compatible with
Python 2, and it takes a while to force yourself to make the jump.

You can use python on the interpreter as a glorified calculator
\begin{lstlisting}[numbers=right]
>>> 5+6
11
>>> 12/9.0
1.3333333333333333
>>> 
\end{lstlisting}
It has big numbers which can be useful sometimes
\begin{lstlisting}[numbers=right]
>>> 2**200
1606938044258990275541962092341162602522202993782792835301376L
\end{lstlisting}
The \texttt{L} at the end is because it is a long number. To get
special functions and so forth you need to \texttt{import} the math
package 
\begin{lstlisting}[numbers=right]
>>> import math
>>> math.tan(math.pi/4.0)
0.9999999999999999
\end{lstlisting}

The syntax for the python math package is more or less the same as for
\texttt{math.h} in C or \texttt{cmath} in C++. Notice there is a namespace,
you need to write \texttt{math.tan(math.pi/4.0)}. If you want to
import the functions and so on into your namespace you use
\texttt{from}, eg
\begin{lstlisting}[numbers=right]
>>> from math import tan 
>>> from math import pi
>>> tan(pi/4.0)
0.9999999999999999
\end{lstlisting}
and, if you want to import everything from \texttt{math} into your current namespace it is 
\begin{lstlisting}[numbers=right]
>>> from math import *
>>> sin(pi)
1.2246467991473532e-16
\end{lstlisting}

Of course, we don't want to use it as an interpreter for anything very
complicated, so we write scripts. However, as an interpreted language
you don't compile it, so if you have a program called \texttt{foo.py}
you write
\begin{lstlisting}[numbers=right]
> python foo.py
\end{lstlisting}
to run it, here the \lq{}$>$\rq{} denotes the command
prompt. Alternatively, you can add a shebang to \texttt{foo.py}, so in
my case I would add \texttt{$\#!$/usr/bin/python} as the first line of
\texttt{foo.py} and change \texttt{foo.py} to executable by typing
\texttt{chmod u+x foo.py} and then I can run \texttt{foo.py} directly
\begin{lstlisting}[numbers=right]
> ./foo.py
\end{lstlisting}
Of course, you might need to change \texttt{/usr/bin/python} to
something else on your machine, it depends on the output of
\texttt{which python}.

\subsection*{Python - some basics}
There are lots of introductions to Python on the web and that's
probably the best place to look. Variables are not declared (\texttt{hello\_world.py})
\begin{lstlisting}[numbers=right]
hello = "hello world"
print hello
\end{lstlisting}
prints hello. It has a nice slice notation  (\texttt{hello\_world\_slice.py})
\begin{lstlisting}[numbers=right]
hello = "hello world"
print hello[0]
print hello[1:4]
print hello[-1]
\end{lstlisting}
prints \texttt{h}, \texttt{ell} and \texttt{d}, \texttt{hello[-1]}
gives the last character. 

One of the most distinctive features is that blocks are created by
indenting rather than brackets. In  (\texttt{hello\_world\_for.py})
\begin{lstlisting}[numbers=right]
hello = "hello world"

for a in hello:
    print a,
print 
\end{lstlisting}
the block for the for loop is \texttt{print a,} because of the indent,
note the colon after the \texttt{for} statement too. This sort of for
loop, looping over parts of an object, is considered more pythonic
than the indexed loops in C++ and so on, this is possible though  (\texttt{hello\_world\_indexed.py})
\begin{lstlisting}[numbers=right]
hello = "hello world"

for i in range(0,len(hello)):
    print hello[i],
print 
\end{lstlisting}
If you do want the index and the object it refers to you should use an
enumeration\\(\texttt{hello\_world\_enumeration.py}):
\begin{lstlisting}[numbers=right]
hello = "hello world"

for i,a in enumerate(hello):
    print i,a
print 
\end{lstlisting}
The if statement is also indented, well all blocks are, but for completeness  (\texttt{if.py})
\begin{lstlisting}[numbers=right]
import random

a=random.randint(1,3)

if a==1:
    print 'one'
elif a==2:
    print 'two'
else:
    print 'three'
\end{lstlisting}


For some reason what are called arrays or vectors in other languages
are called lists in python  (\texttt{if.py})
\begin{lstlisting}[numbers=right]
hello = "hello world"

letters = []

for a in hello:
    letters.append(a)

print letters
\end{lstlisting}
Lists need not be heterogenous  (\texttt{hetro\_list.py})
\begin{lstlisting}[numbers=right]
list=[72,79,86,96,103,"Cathedral Parkway"]
print list
\end{lstlisting}
Python also has a tuple type, like a list but immutable. The
\texttt{enumerate} we saw above makes an \texttt{enumerate} object,
but you can cast it to a list of tuples (\texttt{enumerate\_object.py}):
\begin{lstlisting}[numbers=right]
hello = "hello world"

print enumerate(hello) 

print list(enumerate(hello)) 
\end{lstlisting}

One thing about Python that is hard to get used to if you are used to
C or C++ is that what I keep calling variables are more correctly
labels, they name memory locations rather than pieces of data. This
often catches me out, to see the difference look at (\texttt{labels.py})
\begin{lstlisting}[numbers=right]
a = [0,1,2]
b = a

print a
print b

b[1]=-1

print a
print b
\end{lstlisting}
and note that changing \texttt{b[1]} has changed the value of \texttt{a[1]}.

\subsection*{Python - functional features}

Python has some stylish functional programming feactures for working
with lists, for example \texttt{map} does a function on all the
elements in a sequence so (\texttt{map.py})
\begin{lstlisting}[numbers=right]
def cube(x):
    return x*x*x

numbers = [1,2,3,4,5]

cubes = map (cube,numbers)

print cubes
\end{lstlisting}
where you should also note the syntax for defining functions, there
are functions and classes in the usual way, though classes have no
private members. In fact this code can be made more streamlined, there
is a construction for avoiding giving names to things that don't need
to be named  (\texttt{map\_lambda.py}):
\begin{lstlisting}[numbers=right]
numbers = [1,2,3,4,5]

cubes = map (lambda x:x*x*x,numbers)

print cubes
\end{lstlisting}
In fact, there is also a list comprehension type construction that also does the same thing
\begin{lstlisting}[numbers=right]
numbers = [1,2,3,4,5]

cubes = [x*x*x for x in numbers]

print cubes
\end{lstlisting}

Python contains a number of these functional commands in addition to
\texttt{map} and a number of different data structures, part of the
skill of writing \lq{}pyhtonic\rq{} code is to make these work for
you.

The class construction has a few annoyances; for example each class
can only have one constructor, which is defined using the
\texttt{\_\_init\_\_} function  (\texttt{class\_example.py}).
\begin{lstlisting}[numbers=right]
class Counter:

     def  __init__(self,value):
          self.value=value

     def add(self,increment)
          self.value+=increment

counter=Counter(5)

print counter.value

counter.add(7)

print counter.value

counter.value+=3

print counter.value

\end{lstlisting}
Member variables have the \texttt{self} prefix and notice that if a
class function uses a member variable then \texttt{self} has to be one
of its arguments. Finally, as mentioned before the member variables
are not private.


\subsection*{Scientific calculation packages}

Python has a huge number of packages. For an introduction to some of
them in the context of a very annoying / fun puzzle try
\texttt{http://www.pythonchallenge.com/}. Here we will quickly
introduce packages commonly used in scientific computing 
\begin{itemize}
\item \textsl{Numpy} This adds matrix and vector datatypes and enables
  fast vectorized calculations, it also includes methods for finding
  eigenvectors and eigenvalues.
\item \textsl{Scipy} This adds various numerical routines for working out things like integrals and the solution to differential equations.
\item \textsl{matplotlib} Plots stuff.
\end{itemize}

Here \texttt{numpy} is imported with the name \texttt{np}  (\texttt{np\_array.py})
\begin{lstlisting}[numbers=right]
import numpy as np
d1=np.array([1,-1,1])
d2=np.array([1,2,1])
print d1
print d2
print d1*d2
print np.dot(d1,d2)
\end{lstlisting}
so we define two numpy arrays, we see that \texttt{*} gives element-by-element multiplication, whereas \texttt{np.dot(d1,d2)} gives the dot product. We can do matrix multiplication as well  (\texttt{np\_matrix.py})
\begin{lstlisting}[numbers=right]
import numpy as np
d1=np.array([(1,-1,1),(1,2,1),(1,-1,1)])
print d1
d2=np.array([1,3,1])
print d2
print np.dot(d1,d2)
\end{lstlisting}
and lots of linear algebra  (\texttt{np\_det.py})
\begin{lstlisting}[numbers=right]
import numpy as np
d1=np.array([(1,-1,3),(1,2,1),(3,1,1)])
print np.linalg.det(d1)
\end{lstlisting}

As for \texttt{matplotlib} here is a simple example  (\texttt{plot.py})
\begin{lstlisting}[numbers=right]
import numpy as np
import matplotlib.pyplot as plt

x = np.linspace(0, 10)
plt.plot(x, np.sin(x), linewidth=2)

plt.savefig("example.png")

plt.show()
\end{lstlisting}
It both saves the plot as \texttt{example.png} and shows it on the screen.

\end{document}

